%====================
%- http://nbviewer.ipython.org/github/bokeh/bokeh-notebooks/blob/master/tutorial/00%20-%20intro.ipynb

In [1]:
from IPython.display import display, HTML
import bubble_plot
import pandas as pd

from bokeh.io import show, output_notebook
output_notebook()

from IPython.html.services.config import ConfigManager
from IPython.utils.path import locate_profile
cm = ConfigManager(profile_dir=locate_profile(get_ipython().profile))
cm.update('livereveal', {
              #'width': 1024,
              #'height': 768,
              'theme': 'solarized',
              'transition': 'fade',
              'start_slideshow_at': 'selected',
})
BokehJS successfully loaded.
Out[1]:
{'start_slideshow_at': 'selected', 'theme': 'solarized', 'transition': 'fade'}
 	
Bokeh Tutorial
Sarah Bird
twitter: birdsarah
Bokeh contributor
Bryan Van de Ven
twitter: bryvdv
Lead developer on Bokeh
Getting setup
https://github.com/bokeh/bokeh-notebooks/

\begin{framed}
	\begin{verbatim}
Install Bokeh 0.9.0 & ipython-notebook
conda install bokeh

conda install ipython-notebook

conda install pandas
\end{verbatim}
\end{document}

Python 2 or 3 is fine
Overview of Bokeh, features, ways to use it
%======================================================================= %
Exercises 1 (20 minutes)
01 - plotting.ipynb
02 - charts.ipynb
Navigating the Bokeh docs & examples
Exercises 2 (20 minutes)
03 - models.ipynb
04 - styling.ipynb
 Interactions & sharing plots
 Exercises 3 (20 minutes)
05 - sharing.ipynb
06 - interactions.ipynb

Open space till end of session
%====================================================================================================== %
What is Bokeh
Data visualization library
Build interactive visualizations for the web
Data-driven
Uses canvas (as opposed to svg)
Written in python (or....)
Alternative to d3, matplotlib, ....
In [2]:
display(HTML(bubble_plot.get_bubble_html()))
Why bokeh?
python vs javascript
mid-data (and big-data)
interesting things with the "server" (streaming data, server-side computation)
work in ipython notebook - share interactively on the web
my new default

Python <-> web?
bokeh simple

bokeh pieces

In [3]:
display(HTML(bubble_plot.get_bubble_html()))
Charts interface
Chart functions are awesome!
In [4]:
from bokeh.charts import (
Area, Bar, BoxPlot, Donut, Dot, HeatMap, Histogram,
Horizon, Line, Scatter, Step, TimeSeries, 
)
Get some data
In [5]:
xyvalues = bubble_plot.get_scatter_data()
xyvalues['1964'][0:5]
Out[5]:
[(7.6710000000000003, 33.639000000000003),
 (5.7110000000000003, 65.474999999999994),
 (7.6529999999999996, 47.953000000000003),
 (7.4249999999999998, 34.603999999999999),
 (4.25, 63.774999999999999)]
In [6]:
from bokeh.charts import Scatter
show(Scatter(xyvalues))
	
Basic customizations
In [7]:
scatter = Scatter(
    xyvalues, 
    tools="wheel_zoom, reset", 
    ylabel="Children per woman (total fertility)", 
    xlabel="Life expectancy at birth (years)",
    title="Life expectancy vs Fertility in 1964",
    width=800,
    height=600,
    xgrid=None,
    ygrid=None,
)
scatter.outline_line_color = 'white'
scatter.logo = None
scatter.toolbar_location = 'right'
In [8]:
show(scatter)


Charts interface undergoing major overhaul - check out 0.10 (~end of Summer)
Models
The low-level API that lets you build pieces up individually

In [9]:
from bokeh.models import Range1d, Plot

def get_plot():
    xdr = Range1d(1, 9)
    ydr = Range1d(20, 100)
    plot = Plot(
        x_range=xdr,
        y_range=ydr,
        title="",
        plot_width=800,
        plot_height=400,
        outline_line_color=None,
        toolbar_location=None,
    )
    return plot
show(get_plot())
In [10]:
AXIS_FORMATS = dict(
    minor_tick_in=None,
    minor_tick_out=None,
    major_tick_in=None,
    major_label_text_font_size="10pt",
    major_label_text_font_style="normal",
    axis_label_text_font_size="10pt",

    axis_line_color='#AAAAAA',
    major_tick_line_color='#AAAAAA',
    major_label_text_color='#666666',

    major_tick_line_cap="round",
    axis_line_cap="round",
    axis_line_width=1,
    major_tick_line_width=1,
)
In [11]:
from bokeh.models import LinearAxis, SingleIntervalTicker

def add_axes(plot):
    xaxis = LinearAxis(SingleIntervalTicker(interval=1), axis_label="Children per woman (total fertility)", **AXIS_FORMATS)
    yaxis = LinearAxis(SingleIntervalTicker(interval=20), axis_label="Life expectancy at birth (years)", **AXIS_FORMATS)
    plot.add_layout(xaxis, 'below')
    plot.add_layout(yaxis, 'left')
    return plot
show(add_axes(get_plot()))
In [12]:
from bokeh.models import ColumnDataSource, Text
years, regions, fertility_df, life_expectancy_df, population_df_size, regions_df = bubble_plot._get_data()

text_source = ColumnDataSource({'year': ['1964']})

def add_text(plot):
    plot = add_axes(plot)
    # Add the year in background (add before circle)
    text = Text(x=2, y=35, text='year', text_font_size='150pt', text_color='#EEEEEE')
    plot.add_glyph(text_source, text)
    return plot

show(add_text(get_plot()))
In [13]:
from bokeh.models import Circle, HoverTool
from bokeh.palettes import Spectral6

renderer_source = ColumnDataSource(bubble_plot.get_1964_data())

def add_circles(plot):
    plot = add_text(plot)
    # Add the circle
    circle_glyph = Circle(
        x='fertility', y='life', size='population',
        fill_color='region_color', fill_alpha=0.8,
        line_color='#7c7e71', line_width=0.5, line_alpha=0.5)
    circle_renderer = plot.add_glyph(renderer_source, circle_glyph)

    # Add the hover (only against the circle and not other plot elements)
    tooltips = "@index"
    plot.add_tools(HoverTool(tooltips=tooltips, renderers=[circle_renderer]))
    return plot

show(add_circles(get_plot()))
In [14]:
def add_legend(plot):
    plot = add_circles(plot)
    # Add a custom legend
    text_x = 7
    text_y = 95
    for i, region in enumerate(regions):
        plot.add_glyph(Text(x=text_x, y=text_y, text=[region], text_font_size='10pt', text_color='#666666'))
        plot.add_glyph(Circle(x=text_x - 0.1, y=text_y + 2, fill_color=Spectral6[i], size=10, line_color=None, fill_alpha=0.8))
        text_y = text_y - 5
    return plot
    
show(add_legend(get_plot()))
Interaction
In [19]:
from bokeh.models import CustomJS, Slider
from bokeh.plotting import vplot, hplot
sources = {}

region_color = regions_df['region_color']
region_color.name = 'region_color'

def make_interactive(plot):
    plot = add_legend(plot)
    
    for year in years:
        fertility = fertility_df[year]
        fertility.name = 'fertility'
        life = life_expectancy_df[year]
        life.name = 'life'
        population = population_df_size[year]
        population.name = 'population'
        new_df = pd.concat([fertility, life, population, region_color], axis=1)
        sources['_' + str(year)] = ColumnDataSource(new_df)

    dictionary_of_sources = dict(zip([x for x in years], ['_%s' % x for x in years]))
    js_source_array = str(dictionary_of_sources).replace("'", "")

    # Add the slider
    code = """
        var year = slider.get('value'),
            sources = %s,
            new_source_data = sources[year].get('data');
        renderer_source.set('data', new_source_data);
        renderer_source.trigger('change');
        text_source.set('data', {'year': [String(year)]});
        text_source.trigger('change');
    """ % js_source_array

    callback = CustomJS(args=sources, code=code)
    slider = Slider(start=years[0], end=years[-1], value=1, step=1, title="Year", callback=callback)
    callback.args["slider"] = slider
    callback.args["renderer_source"] = renderer_source
    callback.args["text_source"] = text_source

    
    return vplot(plot, hplot(slider))

show(make_interactive(get_plot()))
In [ ]:
from bokeh.plotting import vplot, hplot
layout = vplot(plot, hplot(slider))
In [ ]:
for year in years:
    fertility = fertility_df[year]
    fertility.name = 'fertility'
    life = life_expectancy_df[year]
    life.name = 'life'
    population = population_df_size[year]
    population.name = 'population'
    new_df = pd.concat([fertility, life, population, region_color], axis=1)
    sources['_' + str(year)] = ColumnDataSource(new_df)
In [16]:
print(sources['_1964'])
print(sources['_1965'])
print(sources['_1966'])
ColumnDataSource, ViewModel:ColumnDataSource, ref _id: db4bbaf4-974d-4538-993c-8b8a1917fc88
ColumnDataSource, ViewModel:ColumnDataSource, ref _id: 186306e5-c65c-4e0a-a7e9-8589997406b0
ColumnDataSource, ViewModel:ColumnDataSource, ref _id: bc5d2dc8-96e7-4f62-9aea-91925ce58f72
In [ ]:
# Add the slider
code = """
    var year = slider.get('value'),
        sources = %s,
        new_source_data = sources[year].get('data');
    renderer_source.set('data', new_source_data);
    renderer_source.trigger('change');
    text_source.set('data', {'year': [String(year)]});
    text_source.trigger('change');
""" % js_source_array

callback = CustomJS(args=sources, code=code)
slider = Slider(start=years[0], end=years[-1], value=1, step=1, title="Year", callback=callback)
callback.args["slider"] = slider
callback.args["renderer_source"] = renderer_source
callback.args["text_source"] = text_source
Embedding
ipython - output_notebook() & show -> works on nbviewer (not github view)
file - output_file() & show / save
embed functions - file_html, components, ...
understand Resources - CDN, INLINE, other...
In [17]:
from jinja2 import Template
from bokeh.templates import RESOURCES
from bokeh.resources import Resources
from bokeh.embed import components

with open('assets/bubble_template.html', 'r') as f:
    template = Template(f.read())

resources = Resources(mode='server', root_url='tree/')

bokeh_js = RESOURCES.render(js_files=resources.js_files)
script, div = components(make_layout(get_plot()))
html = template.render(
    title="Bokeh - Gapminder demo",
    bokeh_js=bokeh_js,
    plot_script=script,
    plot_div=div,
)
In [18]:
display(HTML(html))
\end{document}
%Back to top
%This web site does not host notebooks, it only renders notebooks available on other websites.
%
%Delivered by Fastly, Rendered by Rackspace
%
%nbviewer GitHub repository.
%
%nbviewer version: 03d6df0
%
%IPython version: 4.0.0
%
%Rendered (Sat, 26 Sep 2015 17:27:54 UTC)
