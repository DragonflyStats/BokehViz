Embedding Bokeh Plots
Standalone HTML
Components
IPython Notebook
Autoloading
server data
static data
Bokeh provides a variety of ways to embed plots and data into HTML documents.
%=================================================================================================== %
\section{Standalone HTML}
Bokeh can generate standalone HTML documents using the file_html() function. This function can emit HTML from its own generic template, or a template you provide. These files contain the data for the plot inline and are completely transportable, while still providing interactive tools (pan, zoom, etc.) for your plot. Here is an example:

\begin{framed}
\begin{verbatim}
from bokeh.plotting import figure
from bokeh.resources import CDN
from bokeh.embed import file_html

plot = figure()
plot.circle([1,2], [3,4])

html = file_html(plot, CDN, "my plot")

\end{verbatim}
\end{framed}

The returned HTML text can be saved to a file using standard python file operations.

\subsection{Note}
This is a fairly low-level, explicit way to generate an HTML file. When using the bokeh.plotting or bokeh.charts interfaces, users will typically call the function \texttt{output\_file()} in conjuction with \texttt{show()} or \texttt{save()} instead.
You can also provide your own template and pass in custom, or additional, template variables. See the \texttt{file\_html()} function for more details. You can see an example of this in the gapminder example plot.

%=================================================================================================== %
\section{Components}
It is also possible to ask Bokeh to return the individual components for a inline embedding using the components() function. This function returns a <script> that contains the data for your plot, together with an accompanying <div> tag that the plot view is loaded into. These tags can be used in HTML documents however you like:

\begin{framed}
\begin{verbatim}
from bokeh.plotting import figure
from bokeh.embed import components

plot = figure()
plot.circle([1,2], [3,4])

script, div = components(plot)
\end{verbatim}
\end{framed}
%=================================================================================================== %
The returned <script> will look something like:
\begin{framed}
	\begin{verbatim}
<script type="text/javascript">
    $(function() {
        var modelid = "fba97329-a355-499e-9252-0adc64b19d2e";
        var modeltype = "Plot";
        var elementid = "8ed68feb-d258-4953-9dfb-fb1c13326509";
        Bokeh.logger.info("Realizing plot:")
        Bokeh.logger.info(" - modeltype: Plot");
        Bokeh.logger.info(" - modelid: fba97329-a355-499e-9252-0adc64b19d2e");
        Bokeh.logger.info(" - elementid: 8ed68feb-d258-4953-9dfb-fb1c13326509");

        var all_models = [ JSON PLOT MODELS AND DATA ARE HERE ];

        Bokeh.load_models(all_models);
        var model = Bokeh.Collections(modeltype).get(modelid);
        var view = new model.default_view({
            model: model, el: '#8ed68feb-d258-4953-9dfb-fb1c13326509'
        });
        Bokeh.index[modelid] = view
    });
</script>
\end{verbatim}
\end{framed}
%=================================================================================================== %

All of the data and plot objects are contained in the all_models variable (contents omitted here for brevity). The resulting <div> will look something like:
\begin{verbatim}
<div class="plotdiv" id="8ed68feb-d258-4953-9dfb-fb1c13326509"></div>
\end{verbatim}

These two elements can be inserted or templated into your HTML text, and the script, when executed, will replace the div with the plot.

Using these components assumes that BokehJS has already been loaded, for instance either inline in the document text, or from CDN. To load BokehJS from CDN, add the following lines in your HTML text or template with the appropriate version replacing x.y.z:

%=================================================================================================== %
\begin{framed}
	\begin{verbatim}
<link
    href="http://cdn.pydata.org/bokeh/release/bokeh-x.y.z.min.css"
    rel="stylesheet" type="text/css">
<script src="http://cdn.pydata.org/bokeh/release/bokeh-x.y.z.min.js"></script>
For example, to use version 0.10.0:

<link
    href="http://cdn.pydata.org/bokeh/release/bokeh-0.10.0.min.css"
    rel="stylesheet" type="text/css">
<script src="http://cdn.pydata.org/bokeh/release/bokeh-0.10.0.min.js"></script>
\end{verbatim}
\end{framed}
Note
You must provide the closing </script> tag. This is required by all browsers and the page will typically not render without it.
The \texttt{components()} function takes either a single PlotObject, a list/tuple of PlotObjects, or a dictionary of keys and PlotObjects. Each returns a corresponding data structure of script and div pairs.

The following illustrates how different input types correlate to outputs:

\begin{framed}
\begin{verbatim}

components(plot)
#=> (script, plot_div)

components((plot_1, plot_2))
#=> (script, (plot_1_div, plot_2_div))

components({"Plot 1": plot_1, "Plot 2": plot_2})
#=> (script, {"Plot 1": plot_1_div, "Plot 2": plot_2_div})
Here’s an example of how you would use the multiple plot generator:

# scatter.py

from bokeh.plotting import figure
from bokeh.models import Range1d
from bokeh.embed import components

# create some data
x1 = [0, 1, 2, 3, 4, 5, 6, 7, 8, 9, 10]
y1 = [0, 8, 2, 4, 6, 9, 5, 6, 25, 28, 4, 7]
x2 = [2, 5, 7, 15, 18, 19, 25, 28, 9, 10, 4]
y2 = [2, 4, 6, 9, 15, 18, 0, 8, 2, 25, 28]
x3 = [0, 1, 0, 8, 2, 4, 6, 9, 7, 8, 9]
y3 = [0, 8, 4, 6, 9, 15, 18, 19, 19, 25, 28]

# select the tools we want
TOOLS="pan,wheel_zoom,box_zoom,reset,save"

# the red and blue graphs will share this data range
xr1 = Range1d(start=0, end=30)
yr1 = Range1d(start=0, end=30)

# only the green will use this data range
xr2 = Range1d(start=0, end=30)
yr2 = Range1d(start=0, end=30)

# build our figures
p1 = figure(x_range=xr1, y_range=yr1, tools=TOOLS, plot_width=300, plot_height=300)
p1.scatter(x1, y1, size=12, color="red", alpha=0.5)

p2 = figure(x_range=xr1, y_range=yr1, tools=TOOLS, plot_width=300, plot_height=300)
p2.scatter(x2, y2, size=12, color="blue", alpha=0.5)

p3 = figure(x_range=xr2, y_range=yr2, tools=TOOLS, plot_width=300, plot_height=300)
p3.scatter(x3, y3, size=12, color="green", alpha=0.5)

# plots can be a single PlotObject, a list/tuple, or even a dictionary
plots = {'Red': p1, 'Blue': p2, 'Green': p3}

script, div = components(plots)
print(script)
print(div)
\end{verbatim}
\end{framed}
Running python scatter.py will print out:

<script type="text/javascript">
    Bokeh.$(function() {
        var all_models = [ JSON PLOT MODELS AND DATA ARE HERE ]
        for (idx in plots) {
            var plot = plots[idx]
            var model = Bokeh.Collections(plot.modeltype).get(plot.modelid);
            Bokeh.logger.info('Realizing plot:')
            Bokeh.logger.info(' - modeltype: ' + plot.modeltype);
            Bokeh.logger.info(' - modelid: ' + plot.modelid);
            Bokeh.logger.info(' - elementid: ' + plot.elementid);
            var view = new model.default_view({
                model: model,
                el: plot.elementid
            });
            Bokeh.index[plot.modelid] = view;
        }
    });
</script>

{'Blue': '<div class="plotdiv" id="5fb494f2-e2cb-4eb8-8ec7-11b38143ea30"></div>', 'Green': '<div class="plotdiv" id="f37808b2-e1cc-494e-a126-32f979e2a60d"></div>', 'Red': '<div class="plotdiv" id="a30e4c01-290a-4a19-9b36-c242c53cba8b"></div>'}
Then inserting the script and div elements into this boilerplate:

<!DOCTYPE html>
<html lang="en">
    <head>
        <meta charset="utf-8">
        <title>Bokeh Scatter Plots</title>

        <link rel="stylesheet" href="http://cdn.pydata.org/bokeh/release/bokeh-0.9.0.min.css" type="text/css" />
        <script type="text/javascript" src="http://cdn.pydata.org/bokeh/release/bokeh-0.9.0.min.js"></script>

        <!-- COPY/PASTE SCRIPT HERE -->

    </head>
    <body>
        <!-- INSERT DIVS HERE -->
    </body>
</html>
You can see an example by running:

python /bokeh/examples/embed/embed_multiple.py
%=================================================================================== %
IPython Notebook
Bokeh can also generate <div> tags suitable for inline display in the IPython notebook using the notebook_div() function:

from bokeh.plotting import figure
from bokeh.embed import notebook_div

plot = figure()
plot.circle([1,2], [3,4])

div = notebook_div(plot)
The returned div contains the same sort of <script> and <div> that the components() function above returns.

Note
This is a fairly low-level, explicit way to generate an IPython notebook div. When using the bokeh.plotting or bokeh.charts interfaces, users will typically call the function output_notebook() in conjunction with show() instead.



