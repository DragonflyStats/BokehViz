nbviewer
FAQ
IPython
Jupyter
bokeh-notebooks   tutorial
 	
Bokeh Tutorial — bokeh.models interface
Models
NYTimes interactive chart Usain Bolt vs. 116 years of Olympic sprinters

The first thing we need is to get the data. The data for this chart is located in the bokeh.sampledata module as a Pandas DataFrame. You can see the first ten rows below:?

In [1]:
from bokeh.sampledata.sprint import sprint
sprint[:10]
Out[1]:
Name	Country	Medal	Time	Year
0	Usain Bolt	JAM	GOLD	9.63	2012
1	Yohan Blake	JAM	SILVER	9.75	2012
2	Justin Gatlin	USA	BRONZE	9.79	2012
3	Usain Bolt	JAM	GOLD	9.69	2008
4	Richard Thompson	TRI	SILVER	9.89	2008
5	Walter Dix	USA	BRONZE	9.91	2008
6	Justin Gatlin	USA	GOLD	9.85	2004
7	Francis Obikwelu	POR	SILVER	9.86	2004
8	Maurice Greene	USA	BRONZE	9.87	2004
9	Maurice Greene	USA	GOLD	9.87	2000
Next we import some of the Bokeh models that need to be assembled to make a plot. At a minimum, we need to start with Plot, the glyphs (Circle and Text) we want to display, as well as ColumnDataSource to hold the data and range obejcts to set the plot bounds.

In [2]:
from bokeh.io import output_notebook, show
from bokeh.models.glyphs import Circle, Text
from bokeh.models import ColumnDataSource, Range1d, DataRange1d, Plot
In [3]:
output_notebook()
BokehJS successfully loaded.
Setting up Data
In [4]:
abbrev_to_country = {
    "USA": "United States",
    "GBR": "Britain",
    "JAM": "Jamaica",
    "CAN": "Canada",
    "TRI": "Trinidad and Tobago",
    "AUS": "Australia",
    "GER": "Germany",
    "CUB": "Cuba",
    "NAM": "Namibia",
    "URS": "Soviet Union",
    "BAR": "Barbados",
    "BUL": "Bulgaria",
    "HUN": "Hungary",
    "NED": "Netherlands",
    "NZL": "New Zealand",
    "PAN": "Panama",
    "POR": "Portugal",
    "RSA": "South Africa",
    "EUA": "United Team of Germany",
}

gold_fill   = "#efcf6d"
gold_line   = "#c8a850"
silver_fill = "#cccccc"
silver_line = "#b0b0b1"
bronze_fill = "#c59e8a"
bronze_line = "#98715d"

fill_color = { "gold": gold_fill, "silver": silver_fill, "bronze": bronze_fill }
line_color = { "gold": gold_line, "silver": silver_line, "bronze": bronze_line }

def selected_name(name, medal, year):
    return name if medal == "gold" and year in [1988, 1968, 1936, 1896] else None

t0 = sprint.Time[0]

sprint["Abbrev"]       = sprint.Country
sprint["Country"]      = sprint.Abbrev.map(lambda abbr: abbrev_to_country[abbr])
sprint["Medal"]        = sprint.Medal.map(lambda medal: medal.lower())
sprint["Speed"]        = 100.0/sprint.Time
sprint["MetersBack"]   = 100.0*(1.0 - t0/sprint.Time)
sprint["MedalFill"]    = sprint.Medal.map(lambda medal: fill_color[medal])
sprint["MedalLine"]    = sprint.Medal.map(lambda medal: line_color[medal])
sprint["SelectedName"] = sprint[["Name", "Medal", "Year"]].apply(tuple, axis=1).map(lambda args: selected_name(*args))

source = ColumnDataSource(sprint)
Basic Plot with Glyphs
In [5]:
plot_options = dict(plot_width=800, plot_height=480, toolbar_location=None, 
                    outline_line_color=None, title = "Usain Bolt vs. 116 years of Olympic sprinters")
In [6]:
radius = dict(value=5, units="screen")
medal_glyph = Circle(x="MetersBack", y="Year", radius=radius, fill_color="MedalFill", 
                     line_color="MedalLine", fill_alpha=0.5)

athlete_glyph = Text(x="MetersBack", y="Year", x_offset=10, text="SelectedName",
    text_align="left", text_baseline="middle", text_font_size="9pt")

no_olympics_glyph = Text(x=7.5, y=1942, text=["No Olympics in 1940 or 1944"],
    text_align="center", text_baseline="middle",
    text_font_size="9pt", text_font_style="italic", text_color="silver")
In [7]:
xdr = Range1d(start=sprint.MetersBack.max()+2, end=0)  # +2 is for padding
ydr = DataRange1d(range_padding=0.05)  

plot = Plot(x_range=xdr, y_range=ydr, **plot_options)
plot.add_glyph(source, medal_glyph)
plot.add_glyph(source, athlete_glyph)
plot.add_glyph(no_olympics_glyph)
Out[7]:
<bokeh.models.renderers.GlyphRenderer at 0x1092e3048>
In [8]:
show(plot)
Adding Axes and Grids
In [9]:
from bokeh.models import Grid, LinearAxis, SingleIntervalTicker
In [10]:
xdr = Range1d(start=sprint.MetersBack.max()+2, end=0)  # +2 is for padding
ydr = DataRange1d(range_padding=0.05)  

plot = Plot(x_range=xdr, y_range=ydr, **plot_options)
plot.add_glyph(source, medal_glyph)
plot.add_glyph(source, athlete_glyph)
plot.add_glyph(no_olympics_glyph)
Out[10]:
<bokeh.models.renderers.GlyphRenderer at 0x1092fc470>
In [11]:
xticker = SingleIntervalTicker(interval=5, num_minor_ticks=0)
xaxis = LinearAxis(ticker=xticker, axis_line_color=None, major_tick_line_color=None,
                   axis_label="Meters behind 2012 Bolt", axis_label_text_font_size="10pt", 
                   axis_label_text_font_style="bold")
plot.add_layout(xaxis, "below")

xgrid = Grid(dimension=0, ticker=xaxis.ticker, grid_line_dash="dashed")
plot.add_layout(xgrid)

yticker = SingleIntervalTicker(interval=12, num_minor_ticks=0)
yaxis = LinearAxis(ticker=yticker, major_tick_in=-5, major_tick_out=10)
plot.add_layout(yaxis, "right")
In [12]:
show(plot)
Adding a Hover Tool
In [13]:
from bokeh.models import HoverTool
In [14]:
tooltips = """
<div>
    <span style="font-size: 15px;">@Name</span>&nbsp;
    <span style="font-size: 10px; color: #666;">(@Abbrev)</span>
</div>
<div>
    <span style="font-size: 17px; font-weight: bold;">@Time{0.00}</span>&nbsp;
    <span style="font-size: 10px; color: #666;">@Year</span>
</div>
<div style="font-size: 11px; color: #666;">@{MetersBack}{0.00} meters behind</div>
"""
In [15]:
xdr = Range1d(start=sprint.MetersBack.max()+2, end=0)  # +2 is for padding
ydr = DataRange1d(range_padding=0.05)  

plot = Plot(x_range=xdr, y_range=ydr, **plot_options)
medal = plot.add_glyph(source, medal_glyph)  # we need this renderer to configure the hover tool
plot.add_glyph(source, athlete_glyph)
plot.add_glyph(no_olympics_glyph)

xticker = SingleIntervalTicker(interval=5, num_minor_ticks=0)
xaxis = LinearAxis(ticker=xticker, axis_line_color=None, major_tick_line_color=None,
                   axis_label="Meters behind 2012 Bolt", axis_label_text_font_size="10pt", 
                   axis_label_text_font_style="bold")
plot.add_layout(xaxis, "below")

xgrid = Grid(dimension=0, ticker=xaxis.ticker, grid_line_dash="dashed")
plot.add_layout(xgrid)

yticker = SingleIntervalTicker(interval=12, num_minor_ticks=0)
yaxis = LinearAxis(ticker=yticker, major_tick_in=-5, major_tick_out=10)
plot.add_layout(yaxis, "right")
In [16]:
hover = HoverTool(tooltips=tooltips, renderers=[medal])
plot.add_tools(hover)
In [17]:
show(plot)
In [18]:
from bubble_plot import get_1964_data

def get_plot():
    return Plot(
        x_range=Range1d(1, 9), y_range=Range1d(20, 100),
        title="", plot_width=800, plot_height=400,
        outline_line_color=None, toolbar_location=None,
    )

df = get_1964_data()
df.head()
Out[18]:
fertility	life	population	region_color
Afghanistan	7.671	33.639	9.129985	#fc8d59
Albania	5.711	65.475	3.802632	#e6f598
Algeria	7.653	47.953	9.630513	#fee08b
American Samoa	NaN	NaN	3.000000	#99d594
Andorra	NaN	NaN	3.000000	#e6f598
In [19]:
# EXERCISE: Add Circles to the plot from the data in `df`. 
# With `fertility` for the x coordinates, `life` for the y coordinates.

plot = get_plot()
In [20]:
# EXERCISE: Color the circles by region_color & change the size of the color by population
In [21]:
# EXERCISE: Add axes and grid lines
In [22]:
# EXERCISE: Manually add a legend using Circle & Text. The color key is as follows 

region_name_and_color = [
    ('America', '#3288bd'),
    ('East Asia & Pacific', '#99d594'),
    ('Europe & Central Asia', '#e6f598'),
    ('Middle East & North Africa', '#fee08b'),
    ('South Asia', '#fc8d59'),
    ('Sub-Saharan Africa', '#d53e4f')
]
In [ ]:
 
Back to top
This web site does not host notebooks, it only renders notebooks available on other websites.

Delivered by Fastly, Rendered by Rackspace

nbviewer GitHub repository.

nbviewer version: 03d6df0

IPython version: 4.0.0

Rendered in a few seconds
