
%========================================================================%
\section{Plotting}

Plotting is an intermediate-level interface that is centered around composing visual glyphs. 
Here, you create a visualization by combining various visual elements (dot, circles, line, patch & many others) 
and tools (hover tool, zoom, Save, reset and others).

Bokeh plots created using the bokeh.plotting interface comes with a default set of tools and visual styles. 
For plotting, follow the below steps:

Import library, methods or functions
Select the output mode (notebook, web browser, server)
Activate a figure (similar like matplotlib)
Perform subsequent plotting operations, it will affect the generated figure.
Visualize it
To understand these steps better, let me demonstrate these steps using examples below:


%========================================================================%
 

Plot Example-1: Create a scatter square mark on XY frame of notebook

from bokeh.plotting import figure, output_notebook, show

# output to notebook
output_notebook()
p = figure(plot_width=400, plot_height=400)
# add square with a size, color, and alpha
p.square([2, 5, 6, 4], [2, 3, 2, 1, 2], size=20, color="navy")
# show the results
show(p)



Bokeh_Scatter
Similarly, you can create various other plots like line, wedges & arc, ovals, images, patches and many others, refer this link to see various example.


%========================================================================%
 

Plot Example-2: Combine two visual elements in a plot

from bokeh.plotting import figure, output_notebook, show
# output to notebook
output_notebook()
p = figure(plot_width=400, plot_height=400)
# add square with a size, color, and alpha
p.square([2, 5, 6, 4], [2, 3, 2, 1, 2], size=20, color="navy")
p.line([1, 2, 3, 4, 5], [1, 2, 2, 4, 5], line_width=2) #added a line plot to existing figure
# show the results
show(p)
Multiple_Plots


%========================================================================%

Plot Example-3: Add a hover tool and axis labels to above plot

from bokeh.plotting import figure, output_notebook, show
from bokeh.models import HoverTool, BoxSelectTool #For enabling tools
# output to notebook
output_notebook()
#Add tools
TOOLS = [BoxSelectTool(), HoverTool()]
p = figure(plot_width=400, plot_height=400, tools=TOOLS)
# add a square with a size, color, and alpha
p.square([2, 5, 6, 4], [2, 3, 2, 1, 2], size=20, color="navy", alpha=0.5)
#Visual Elements
p.xaxis.axis_label = "X-axis"
p.yaxis.axis_label = "Y-axis"
# show the results
show(p)
Bokeh_Tools_Visualize

For more details on visual attributes and tools refer these links:

Styling visual attributes
Configuring plot tools
 

Plot Example-4: Plot map of India using latitude and longitude data for boundaries


%========================================================================%


Note: I have data for polygon of latitude and longitude for boundaries of India in a csv format. I will use that for plotting.

Here, we will go with patch plotting, let’s look at the commands below:

#Import libraries
import pandas as pd
from bokeh.plotting import figure, show, output_notebook
#Import Latitude and lanogitude co-ordinates
India=pd.read_csv('E:/India.csv')
del India['ID']
India.index=['IN0','IN1','IN2','IN3','IN4','IN5']
#Convert string values to float as co-ordinates in dataframe are string
for j in range(0,len(India)):
 a = India['lats'][j]
 India['lats'][j] = [float(i) for i in a[1:len(a)-1].split(",")]
for j in range(0,len(India)):
 a = India['lons'][j]
 India['lons'][j] = [float(i) for i in a[1:len(a)-1].split(",")]
# Output option
output_notebook()
# Create your plot
p = figure(plot_height=400, plot_width=400, toolbar_location="right",x_axis_type=None, y_axis_type=None)
p.patches(xs=India['lons'], ys=India['lats'], fill_color="white",line_color="black", line_width=0.5)
#Visualize your chart
show(p)
INDIA
