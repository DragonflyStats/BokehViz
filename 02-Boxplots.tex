\begin{document}
Box Plots
The BoxPlot can be used to summarize the statistical properties of different groups of data. The label specifies a column in the data to group by, and a box plot is generated for each group:
\begin{framed}
	\begin{verbatim}
from bokeh.charts import BoxPlot, output_file, show
from bokeh.sampledata.autompg import autompg as df

p = BoxPlot(df, values='mpg', label='cyl',
title="MPG Summary (grouped by CYL)")

output_file("boxplot.html")

show(p)
\end{verbatim}
\end{framed}
The label can also accept a list of column names, in which case the data is grouped by all the groups in the list:
\begin{framed}
	\begin{verbatim}
from bokeh.charts import BoxPlot, output_file, show
from bokeh.sampledata.autompg import autompg as df

p = BoxPlot(df, values='mpg', label=['cyl', 'origin'],
title="MPG Summary (grouped by CYL, ORIGIN)")

output_file("boxplot.html")

show(p)
\end{verbatim}
\end{framed}
%========================================================================= %
\subsection{Box Color}
The color of the box in a BoxPlot can be set to a fixed color using the color parameter:

\begin{framed}
	\begin{verbatim}
from bokeh.charts import BoxPlot, output_file, show
from bokeh.sampledata.autompg import autompg as df

p = BoxPlot(df, values='mpg', label='cyl', color='#00cccc',
title="MPG Summary (grouped by CYL)")

output_file("boxplot.html")

show(p)
\end{verbatim}
\end{framed}
As with Bar charts, the color can also be given a column name, in which case the boxes are shaded automatically according to the group:
\begin{framed}
	\begin{verbatim}
from bokeh.charts import BoxPlot, output_file, show
from bokeh.sampledata.autompg import autompg as df

p = BoxPlot(df, values='mpg', label='cyl', color='cyl',
title="MPG Summary (grouped and shaded by CYL)")

output_file("boxplot.html")

show(p)
\end{verbatim}
\end{framed}
\subsection{Whisker Color}
The color of the whiskers can be similary controlled using the whisker_color paramter. For a single color:
\begin{framed}
\begin{verbatim}
from bokeh.charts import BoxPlot, output_file, show
from bokeh.sampledata.autompg import autompg as df

p = BoxPlot(df, values='mpg', label='cyl', whisker_color='goldenrod',
title="MPG Summary (grouped by CYL, shaded whiskers)")

output_file("boxplot.html")

show(p)
\end{verbatim}
\end{framed}
Or shaded automatically according to a column grouping:

\begin{framed}
\begin{verbatim}
from bokeh.charts import BoxPlot, output_file, show
from bokeh.sampledata.autompg import autompg as df

p = BoxPlot(df, values='mpg', label='cyl', whisker_color='cyl',
title="MPG Summary (grouped and whiskers shaded by CYL)")

output_file("boxplot.html")

show(p)
\end{verbatim}
\end{framed}
%==================================================================== %
\subsection{Outliers}
\begin{itemize}
\item By default, BoxPlot charts show outliers above and below the whiskers. 
\item However, the display of outliers can be turned on or off with the outliers parameter:
\end{itemize}


\begin{verbatim}
from bokeh.charts import BoxPlot, output_file, show
from bokeh.sampledata.autompg import autompg as df

p = BoxPlot(df, values='mpg', label='cyl', outliers=False,
title="MPG Summary (grouped by CYL, no outliers)")

output_file("boxplot.html")

show(p)
\end{verbatim}


%==================================================================== %
\subsection{Markers}
The marker used for displaying outliers is controlled by the marker parameter:

from bokeh.charts import BoxPlot, output_file, show
from bokeh.sampledata.autompg import autompg as df

p = BoxPlot(df, values='mpg', label='cyl', marker='square',
title="MPG Summary (grouped by CYL, square marker)")

output_file("boxplot.html")

show(p)

H

\end{document}