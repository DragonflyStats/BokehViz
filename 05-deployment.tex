\documentclass[a4paper,12pt]{article}
%%%%%%%%%%%%%%%%%%%%%%%%%%%%%%%%%%%%%%%%%%%%%%%%%%%%%%%%%%%%%%%%%%%%%%%%%%%%%%%%%%%%%%%%%%%%%%%%%%%%%%%%%%%%%%%%%%%%%%%%%%%%%%%%%%%%%%%%%%%%%%%%%%%%%%%%%%%%%%%%%%%%%%%%%%%%%%%%%%%%%%%%%%%%%%%%%%%%%%%%%%%%%%%%%%%%%%%%%%%%%%%%%%%%%%%%%%%%%%%%%%%%%%%%%%%%
\usepackage{eurosym}
\usepackage{vmargin}
\usepackage{amsmath}
\usepackage{graphics}
\usepackage{epsfig}
\usepackage{subfigure}
\usepackage{fancyhdr}
\usepackage{listings}
\usepackage{framed}
\usepackage{graphicx}
\usepackage{amsmath}
\usepackage{chngpage}
%\usepackage{bigints}

\setcounter{MaxMatrixCols}{10}

\begin{document}
\section{Bokeh Tutorial — Sharing and Embedding}
%=============================================================== %
\subsection{Creating a Plot to Share}
\begin{framed}
\begin{verbatim}
import pandas as pd
from bokeh.plotting import figure

AAPL = pd.read_csv(
"http://ichart.yahoo.com/table.csv?s=AAPL&a=0&b=1&c=2000&d=0&e=1&f=2015",
parse_dates=['Date'])

MSFT = pd.read_csv(
"http://ichart.yahoo.com/table.csv?s=MSFT&a=0&b=1&c=2000&d=0&e=1&f=2015",
parse_dates=['Date'])

IBM = pd.read_csv(
"http://ichart.yahoo.com/table.csv?s=IBM&a=0&b=1&c=2000&d=0&e=1&f=2015",
parse_dates=['Date'])
\end{verbatim}
\end{framed}

%================================================= %
\begin{framed}
	\begin{verbatim}
def make_figure():
p = figure(x_axis_type="datetime", width=700, height=300)

p.line(AAPL['Date'], AAPL['Adj Close'], color='#A6CEE3', legend='AAPL')
p.line(IBM['Date'], IBM['Adj Close'], color='#33A02C', legend='IBM')
p.line(MSFT['Date'], MSFT['Adj Close'], color='#FB9A99', legend='MSFT')

p.title = "Stock Closing Prices"
p.grid.grid_line_alpha=0.3
p.xaxis.axis_label = 'Date'
p.yaxis.axis_label = 'Price'
p.legend.orientation = "top_left"
return p
\end{verbatim}
\end{framed}

Embed functions all look for a Plot object. Common plot objects are:

Plot from models
figure from plotting
Chart from charts
vplot, hplot from plotting
Displaying in the Notebook

\begin{framed}
\begin{verbatim}
In [2]:
from bokeh.io import output_notebook, show
In [3]:
output_notebook()
\end{verbatim}
\end{framed}
BokehJS successfully loaded.
\begin{framed}
\begin{verbatim}
In [4]:
p = make_figure()
show(p)
\end{verbatim}
\end{framed}
Saving to an HTML File
\begin{verbatim}
In [5]:
from bokeh.io import output_file, show
In [6]:
output_file("tutorial_sharing.html")
In [7]:
p = make_figure()
show(p)   # save(p) will save without opening a new browser tab
\end{verbatim}

Templating in HTML Documents
In [8]:
import jinja2
from bokeh.embed import components
\begin{verbatim}
</body>

</html>
""")
In [10]:
p = make_figure()
script, div = components(p)
In [11]:
from IPython.display import HTML
HTML(template.render(script=script, div=div))
Out[11]:
Hello Bokeh!
Below is a simple plot of stock closing prices

	
In [12]:
from flask import Flask
app = Flask(__name__)

@app.route('/')
def hello_bokeh():
    return template.render(script=script, div=div)
In [13]:
# Uncomment to run the Flask Server. Use Kernel -> Interrupt from Notebook menubar to stop 
#app.run(port=5050)
In [14]:
# EXERCISE: Create your own template (or modify the one above) 
In [ ]:
 In [9]:
 template = jinja2.Template("""
 <!DOCTYPE html>
 <html lang="en-US">
 
 <link
 href="http://cdn.pydata.org/bokeh/release/bokeh-0.9.0.min.css"
 rel="stylesheet" type="text/css"
 >
 <script 
 src="http://cdn.pydata.org/bokeh/release/bokeh-0.9.0.min.js"
 ></script>
 
 <body>
 
 <h1>Hello Bokeh!</h1>
 
 <p> Below is a simple plot of stock closing prices </p>
 
 {{ script }}
 
 {{ div }}
\end{verbatim}
%Back to top
%This web site does not host notebooks, it only renders notebooks available on other websites.
%
%Delivered by Fastly, Rendered by Rackspace
%
%nbviewer GitHub repository.
%
%nbviewer version: 03d6df0
%
%IPython version: 4.0.0
%
%Rendered (Sat, 26 Sep 2015 17:34:18 UTC)
\end{document}