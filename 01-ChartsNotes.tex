%%- http://www.analyticsvidhya.com/blog/2015/08/interactive-data-visualization-library-python-bokeh/

 

\section{Visualization with Bokeh}

Bokeh offers both powerful and flexible features which imparts simplicity and highly advanced customization. 



It provides multiple visualization interfaces to the user as shown below:\texttt{Bokeh\_Interface}


\begin{description}
\item[Charts:] a high-level interface that is used to build complex statistical plots as quickly and in a simplistic manner.
\item[Plotting:] an intermediate-level interface that is centered around composing visual glyphs.
\item[Models:] a low-level interface that provides the maximum flexibility to application developers.
\end{description}
In this article, we will look at first two interfaces charts & plotting only. 

We will discuss models and other advance feature of this library in next post.


%========================================================================% 

\section{Charts}

As mentioned above, it is a high level interface used to present information in standard visualization form. 
These forms include box plot, bar chart, area plot, heat map, donut chart and many others. You can generate these plots just by passing data frames, numpy arrays and dictionaries.

Let’s look at the common methodology to create a chart:

\begin{itemize}
\item Import the library and functions/ methods
\item Prepare the data
\item Set the output mode (Notebook, Web Browser or Server)
\item Create chart with styling option (if required)
\item Visualize the chart
\end{itemize}


To understand these steps better, let me demonstrate these steps using example below:

Charts Example-1: Create a bar chart and visualize it on web browser using Bokeh

We will follow above listed steps to create a chart:

\begin{framed}
\begin{verbatim}
#Import library
from bokeh.charts import Bar, output_file, show #use output_notebook to visualize it in notebook
# prepare data (dummy data)
data = {"y": [1, 2, 3, 4, 5]}
# Output to Line.HTML
output_file("lines.html", title="line plot example") #put output_notebook() for notebook
# create a new line chat with a title and axis labels
p = Bar(data, title="Line Chart Example", xlabel='x', ylabel='values', width=400, height=400)
# show the results
show(p)
Bar_Chart
\end{verbatim}
\end{framed}
In the chart above,  you can see the tools at the top (zoom, resize, reset, wheel zoom) and these 
tools allows you to interact with chart. 
You can also look at the multiple chart options (legend, xlabel, ylabel, xgrid, width, height and many other) and 
various example of charts here.


%========================================================================% 

Chart Example-2: Compare the distribution of sepal length and petal length of IRIS data set using 
Box plot on notebook

To create this visualization, firstly, I’ll import the iris data set using sklearn library. Then, 
follow the steps as discussed above to visualize chart in ipython notebook.

\begin{framed}
\begin{verbatim}
#IRIS Data Set

from sklearn.datasets import load_iris
import pandas as pd
iris = load_iris()
df=pd.DataFrame(iris.data)
df.columns=['petal_width','petal_length','sepal_width','sepal_length']

#Import library
from bokeh.charts import BoxPlot, output_notebook, show
data=df[['petal_length','sepal_length']]

# Output to Notebook
output_notebook()
# create a new line chat with a title and axis labels
p = BoxPlot(data, width=400, height=400)

# show the results
show(p)
Bokeh_Box_Plot
\end{verbatim}
\end{framed}
%========================================================================%
 

Chart Example-3: Create a line plot to bokeh server

Prior to plotting visualization to Bokeh server, you need to run it.

If you are using a conda package, you can use run command bokeh-server from any directory 
using command. Else, \texttt{python ./bokeh-server} command should work in general. For more detail on this 
please refer this link “Deploying Bokeh Server“.

There are multiple benefits of Plotting visualization on Bokeh server:

%================================================================== %
\begin{itemize}
\item Plots can be published to larger audience
\item Visualize large data set interactively
\item Streaming data to automatically updating plots
\item Building dashboards and apps
\end{itemize}
%================================================================== %
To start plotting on Bokeh server, I have executed the command bokeh-server to initialize it followed by 
the commands used for visualization.
\begin{framed}
	\begin{verbatim}
Bokeh_Server

from bokeh.plotting import figure, output_server, show
output_server("line")
p = figure(plot_width=400, plot_height=400)
# add a line renderer
p.line([5, 2, 3, 4, 5], [5, 7, 2, 4, 5], line_width=2)
show(p)
Bokeh_Server_Visualization
\end{verbatim}
\end{framed}

\end{document}