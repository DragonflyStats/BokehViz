\documentclass[a4paper,12pt]{article}
%%%%%%%%%%%%%%%%%%%%%%%%%%%%%%%%%%%%%%%%%%%%%%%%%%%%%%%%%%%%%%%%%%%%%%%%%%%%%%%%%%%%%%%%%%%%%%%%%%%%%%%%%%%%%%%%%%%%%%%%%%%%%%%%%%%%%%%%%%%%%%%%%%%%%%%%%%%%%%%%%%%%%%%%%%%%%%%%%%%%%%%%%%%%%%%%%%%%%%%%%%%%%%%%%%%%%%%%%%%%%%%%%%%%%%%%%%%%%%%%%%%%%%%%%%%%
\usepackage{eurosym}
\usepackage{vmargin}
\usepackage{amsmath}
\usepackage{graphics}
\usepackage{epsfig}
\usepackage{subfigure}
\usepackage{fancyhdr}
\usepackage{listings}
\usepackage{framed}
\usepackage{graphicx}
\usepackage{amsmath}
\usepackage{chngpage}
%\usepackage{bigints}


\setcounter{MaxMatrixCols}{10}

\begin{document}
\large
%Defining Key Concepts
%Glossary
%Interfaces
%bokeh.models
%bokeh.plotting
%bokeh.charts
%other interfaces
%

\section*{Glossary}
In order to make the best use of this User Guide, it is important to have context for some high level concepts and terms. Here is a small glossary of some of the most important concepts in Bokeh.

\subsection*{BokehJS}
The JavaScript client library that actually renders the visuals and handles the UI interactions for Bokeh plots and widgets in the browser. Typically, users will not have to think about this aspect of Bokeh much (“We write the JavaScript, so you don’t have to!”) but it is good to have basic knowledge of this dichotomy. For full details, see the BokehJS chapter of the Developer Guide.
\subsection*{Charts}
Schematic statistical plots such as bar charts, horizon plots, time series, etc. that may include faceting, grouping, or stacking based on the structure of the data. Bokeh provides a high level bokeh.charts interface to quickly construct these kinds of plots. See Using High-level Charts for examples and usage.
\subsection*{Embedding}
Various methods of including Bokeh plots and widgets into web apps and pages, or the IPython notebook. 
% See Embedding Bokeh Plots for more details.
\subsection*{Glyphs}
The basic visual building blocks of Bokeh plots, e.g. lines, rectangles, squares, wedges, patches, etc. The \texttt{bokeh.plotting} interface provides a convenient way to create plots centered around glyphs. 
% See Plotting with Basic Glyphs for more information.
\subsection*{Models}
The lowest-level objects that comprise Bokeh “scenegraphs”. These live in the \texttt{bokeh.models} interface. Most users will not use this level of interface to assemble plots directly. 

However, ultimately all Bokeh plots consist of collections of models, so it is important to understand them enough to configure their attributes and properties. 
% See Styling Visual Attributes for more information.

\subsection*{Server}
The bokeh-server is an optional component that can be used for sharing and publishing Bokeh plots and apps, for handling streaming of large data sets, or for enabling sophisticated user interactions based off of widgets and selections. 
% See Deploying the Bokeh Server for more explanation.

\subsection*{Widgets}
User interface elements outside of a Bokeh plot such as sliders, drop down menus, buttons, etc. Events and updates from widgets can inform additional computations, or cause Bokeh plots to update.
% See Adding Interactions for examples and information.

\end{document}