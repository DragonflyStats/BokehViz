\documentclass[a4paper,12pt]{article}
%%%%%%%%%%%%%%%%%%%%%%%%%%%%%%%%%%%%%%%%%%%%%%%%%%%%%%%%%%%%%%%%%%%%%%%%%%%%%%%%%%%%%%%%%%%%%%%%%%%%%%%%%%%%%%%%%%%%%%%%%%%%%%%%%%%%%%%%%%%%%%%%%%%%%%%%%%%%%%%%%%%%%%%%%%%%%%%%%%%%%%%%%%%%%%%%%%%%%%%%%%%%%%%%%%%%%%%%%%%%%%%%%%%%%%%%%%%%%%%%%%%%%%%%%%%%
\usepackage{eurosym}
\usepackage{vmargin}
\usepackage{amsmath}
\usepackage{graphics}
\usepackage{epsfig}
\usepackage{subfigure}
\usepackage{fancyhdr}
\usepackage{listings}
\usepackage{framed}
\usepackage{graphicx}
\usepackage{amsmath}
\usepackage{chngpage}
%\usepackage{bigints}

\setcounter{MaxMatrixCols}{10}

\begin{document}
% % Where Does This Come From

\section{Scatter Plots}
\begin{itemize}
\item The \texttt{Scatter} high-level chart can be used to generate 1D or (more commonly) 2D scatter plots. 
\item It is used by passing in DataFrame-like object as the first argument then specifying the columns to use for x and y coordinates:
\end{itemize}


\begin{framed}
\begin{verbatim}
from bokeh.charts import Scatter, output_file, show
from bokeh.sampledata.autompg import autompg as df

p = Scatter(df, x='mpg', y='hp', 
            title="HP vs MPG",
            xlabel="Miles Per Gallon", ylabel="Horsepower")

output_file("scatter.html")

show(p)
\end{verbatim}
\end{framed}

%================================================================================== %

\subsection{Color}
The \texttt{color} parameter can be used to control the color of the scatter markers:

\begin{framed}
\begin{verbatim}
	
from bokeh.charts import Scatter, output_file, show
from bokeh.sampledata.autompg import autompg as df

p = Scatter(df, x='mpg', y='hp', 
            title="HP vs MPG", 
            color="navy",   
            xlabel="Miles Per Gallon",
            ylabel="Horsepower")

output_file("scatter.html")

show(p)
\end{verbatim}
\end{framed}

%================================================================================== %
\newpage
\subsection{Color Groups}
If color is supplied with the name of a data column then the data is first grouped by the values of that column, and then a different color is used for every group:

\begin{framed}
	\begin{verbatim}
	
from bokeh.charts import Scatter, output_file, show
from bokeh.sampledata.autompg import autompg as df

p = Scatter(df, x='mpg', y='hp', 
            color='cyl',   #  <<<---------------NEW
            title="HP vs MPG (shaded by CYL)",
            xlabel="Miles Per Gallon",
            ylabel="Horsepower")

output_file("scatter.html")

show(p)
\end{verbatim}
\end{framed}
%================================================================================== %

\subsection{Adding Legends}
When grouping, a legend is usually useful, and it’s location can be specified by the legend parameter:
\begin{framed}
	\begin{verbatim}
	
from bokeh.charts import Scatter, output_file, show
from bokeh.sampledata.autompg import autompg as df

p = Scatter(df, x='displ', y='hp', 
            color='cyl',
            title="HP vs DISPL (shaded by CYL)", legend="top_left",
            xlabel="Displacement", ylabel="Horsepower")

output_file("scatter.html")

show(p)
\end{verbatim}
\end{framed}
%================================================================================== %
\newpage
\subsection{Markers}
The marker parameter can be used to control the shape of the scatter marker:
\begin{framed}
\begin{verbatim}

from bokeh.charts import Scatter, output_file, show
from bokeh.sampledata.autompg import autompg as df

p = Scatter(df, x='displ', y='hp', 
            marker='square',
            title="HP vs DISPL", 
            legend="top_left",
            xlabel="Displacement", ylabel="Horsepower")

output_file("scatter.html")

show(p)
\end{verbatim}
\end{framed}
%================================================================================== %

As with color, the \texttt{marker} parameter can be given a column name to group by the values of that column, using a different marker shape for each group:
\begin{framed}
\begin{verbatim}

from bokeh.charts import Scatter, output_file, show
from bokeh.sampledata.autompg import autompg as df

p = Scatter(df, x='displ', y='hp', 
            marker='cyl',   
            title="HP vs DISPL (marked by CYL)",
            legend="top_left",
            xlabel="Displacement",
            ylabel="Horsepower")

output_file("scatter.html")

show(p)
\end{verbatim}
\end{framed}
%================================================================================== %
	
Often it is most useful to group both the color and marker shape together:
\begin{framed}
\begin{verbatim}

from bokeh.charts import Scatter, output_file, show
from bokeh.sampledata.autompg import autompg as df

p = Scatter(df, x='displ', y='hp', 
            marker='cyl', color='cyl',
            title="HP vs DISPL (marked by CYL)", legend="top_left",
            xlabel="Displacement", ylabel="Horsepower")

output_file("scatter.html")

show(p)
\end{verbatim}
\end{framed}
%================================================================================== %
\end{document}